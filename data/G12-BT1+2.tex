\tikzstyle{startstop} = [rectangle, rounded corners, minimum width=10cm, minimum height=1.5cm,text centered, draw=black, fill=green!20]
\begin{center}
	\begin{tikzpicture}
		\node (start) [startstop] {\bfseries \text{ÔN TẬP BÀI 1 VÀ BÀI 2}};
	\end{tikzpicture}
\end{center}
\setcounter{section}{0}
\section{Câu trắc nghiệm nhiều phương án lựa chọn}
\Opensolutionfile{ans}[ans/G12BT1+2TN]
% ===================================================================
\begin{ex}
	Với mô hình động học phân tử, sự khác biệt về cấu trúc của chất rắn, chất lỏng, chất khí là do sự khác biệt về
	\choice
	{thành phần các phân tử cấu tạo của mỗi chất}
	{\True độ lớn lực tương tác giữa các phân tử trong mỗi chất}
	{số lượng phân tử cấu tạo nên mỗi chất}
	{kích thước của các phân tử cấu tạo mỗi chất}
	\loigiai{}
\end{ex}
% ===================================================================
\begin{ex}
Khi nói về khoảng cách trung bình giữa các phân tử trong chất rắn, chất lỏng, chất khí. Kết luận nào sau đây là \textbf{đúng}?	
	\choice
	{Khoảng cách giữa các phân tử trong chất lỏng xa hơn so với các phân tử trong chất khí}
	{Khoảng cách giữa các phân tử trong chất rắn xa hơn so với các phân tử trong chất lỏng}
	{\True Khoảng cách giữa các phân tử trong chất lỏng gần hơn so với các phân tử trong chất khí}
	{Khoảng cách giữa các phân tử trong chất lỏng xa hơn so với các phân tử trong chất khí}
	\loigiai{}
\end{ex}
% ===================================================================
\begin{ex}
	Câu nào dưới đây nói về đặc tính của chất rắn kết tinh là \textbf{không đúng}?
	\choice
	{Các nguyên tử, phân tử liên kết chặt với nhau và sắp xếp theo một trật tự hình học xác định}
	{\True Không có nhiệt độ nóng chảy xác định}
	{Có cấu trúc tinh thể}
	{Có nhiệt độ nóng chảy xác định}
	\loigiai{}
\end{ex}
% ===================================================================
\begin{ex}
	Chất rắn nào dưới đây thuộc loại chất rắn vô định hình?
	\choice
	{Muối ăn}
	{Nhựa đường}
	{Kim loại}
	{Kim cương}
	\loigiai{}
\end{ex}
% ===================================================================
\begin{ex}
	Chất lỏng không có hình dạng xác định vì các phân tử chất lỏng
	\choice
	{dao động tại các vị trí cân bằng xác định}
	{có thể chuyển động phân tán ra xa nhau}
	{\True dao động quanh các vị trí cân bằng có thể dịch chuyển được}
	{có thể chuyển động tự do}
	\loigiai{}
\end{ex}
% ===================================================================
\begin{ex}
Chất khí dễ bị nén hơn so với chất rắn và chất lỏng vì	
	\choice
	{lực tương tác giữa các phân tử trong chất khí lớn hơn so với lực tương tác giữa các phân tử trong chất rắn và chất lỏng}
	{\True khoảng cách giữa các phân tử trong chất khí lớn hơn so với khoảng cách giữa các phân tử trong chất rắn và chất lỏng}
	{các phân tử trong chất khí ít chuyển động hơn so với các phân tử trong chất rắn và chất lỏng}
	{các phân tử trong chất khí có kích thước nhỏ hơn so với các phân tử trong chất rắn và chất lỏng}
	\loigiai{}
\end{ex}
% ===================================================================
\begin{ex}
Khi nói về quá trình nóng chảy và đông đặc là đang nói về quá trình chuyển thể giữa	
	\choice
	{chất rắn và chất khí}
	{chất khí và chất lỏng}
	{\True chất rắn và chất lỏng}
	{các chất bất kì}
	\loigiai{}
\end{ex}
% ===================================================================
\begin{ex}
	Quá trình chuyển từ thể khí sang thể rắn của các chất được gọi là
	\choice
	{\True sự ngưng kết}
	{thăng hoa}
	{sự đông đặc}
	{sự ngưng tụ}
	\loigiai{}
\end{ex}
% ===================================================================
\begin{ex}
	Khi chất rắn kết tinh được nung nóng. Kết luận nào sau đây là \textbf{đúng}?
	\choice
	{các phân tử vẫn dao động với biên độ không đổi, khoảng cách giữa các phân tử không đổi}
	{\True các phân tử dao động với biên độ tăng lên, khoảng cách giữa các phân tử tăng lên}
	{các phân tử dao động với biên độ không đổi, khoảng cách giữa các phân tử tăng lên}
	{các phân tử dao động với biên độ tăng lên, khoảng cách giữa các phân tử không đổi}
	\loigiai{}
\end{ex}
% ===================================================================
\begin{ex}
	Sự nóng chảy của chất rắn kết tinh bắt đầu xảy ra khi
	\choice
	{một số phân tử dao động mạnh hơn các phân tử xung quanh}
	{một số phân tử va chạm với các phân tử xung quanh}
	{một số phân tử dao động mạnh lên và truyền năng lượng dao động cho các phân tử khác}
	{\True một số phân tử thắng được lực liên kết với các phân tử xung quanh và thoát khỏi liên kết với chúng}
	\loigiai{}
\end{ex}
% ===================================================================
\begin{ex}
Trong quá trình chất rắn kết tinh đang nóng chảy nhiệt độ của nó không tăng thêm là do
	\choice
	{phần nhiệt nhận thêm cân bằng với phần nhiệt toả ra môi trường bên ngoài}
	{phần nhiệt lượng nhận thêm đã chuyển thành động năng của các phân tử}
	{\True phần nhiệt lượng nhận thêm đã chuyển thành năng lượng để tiếp tục phá vỡ liên kết của mạng tinh thể}
	{phần nhiệt lượng nhận thêm đã chuyển thành thế năng của các phân tử}
	\loigiai{}
\end{ex}
% ===================================================================
\begin{ex}
	Khi cho một cục nước đá vào nước ở nhiệt độ phòng thì kết luận nào sau đây là \textbf{đúng}?
	\choice
	{Nhiệt độ của nước trong cốc từ từ tăng lên}
	{Nước trong cốc sẽ nhận nhiệt lượng từ cục nước đá}
	{\True Nhiệt lượng được truyền từ nước trong cốc cho cục nước đá}
	{Quá trình truyền nhiệt kết thúc khi cục nước đá tan hết.}
	\loigiai{}
\end{ex}
% ===================================================================
\begin{ex}
	Có ba vật A, B, C có các nhiệt độ lần lượt là $t_\text{A}$, $t_\text{B}$, $t_\text{C}$. Cho vật A tiếp xúc với vật B đến khi cân bằng nhiệt, ngay sau đó lại cho vật A tiếp xúc với vật C đến khi cân bằng nhiệt thì nhiệt độ của vật A lúc này bằng với nhiệt độ của nó lúc ban đầu khi chưa tiếp xúc với các vật khác. Kết luận nào sau đây là \textbf{đúng}?
	\choice
	{Vật B đóng vai trò truyền nhiệt lượng khi tiếp xúc với vật A}
	{Vật C đóng vai trò nhận nhiệt lượng khi tiếp xúc với vật A}
	{Nhiệt độ của vật B thấp hơn nhiệt độ của vật C}
	{\True Tổng nhiệt lượng mà vật A nhận được bằng tổng nhiệt lượng mà nó truyền cho vật khác}
	\loigiai{}
\end{ex}
% ===================================================================
\begin{ex}
	Gọi $t_1$, $t_2$ lần lượt là nhiệt độ điểm đóng băng và nhiệt độ sôi của nước tinh khiết ở điều kiện áp suất tiêu chuẩn. Trong thang nhiệt độ Celsius, mỗi độ chia $\left(\SI{1}{\celsius}\right)$ có độ lớn bằng
	\choice
	{$100\left(t_2-t_1\right)$}
	{$100\left(t_1-t_2\right)$}
	{\True $\dfrac{1}{100}\left(t_2-t_1\right)$}
	{$\dfrac{1}{273,15}\left(t_2-t_1\right)$}
	\loigiai{}
\end{ex}
% ===================================================================
\begin{ex}
Nhiệt độ không tuyệt đối là nhiệt độ mà tại đó tất cả các chất có	
	\choice
	{động năng chuyển động nhiệt của các nguyên tử hoặc phân tử bằng không và thế năng của chúng là cực đại}
	{động năng chuyển động nhiệt của các nguyên tử hoặc phân tử là cực đại và thế năng của chúng là tối thiểu}
	{\True động năng chuyển động nhiệt của các nguyên tử hoặc phân tử bằng không và thế năng của chúng là tối thiểu}
	{động năng chuyển động nhiệt của các nguyên tử hoặc phân tử và thế năng của chúng là cực đại}
	\loigiai{}
\end{ex}
\Closesolutionfile{ans}